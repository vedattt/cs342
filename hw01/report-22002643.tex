\documentclass[a4paper, 12pt, titlepage]{article}

% Document quality things
\usepackage[utf8]{inputenc}
\usepackage{microtype}
\usepackage[dvipsnames]{xcolor}
\usepackage{csquotes, verbatim, listings}
\usepackage{url, hyperref}
\hypersetup{colorlinks=true, linkcolor=black, citecolor=black, urlcolor=blue}

% Image-related packages
\usepackage{graphicx}
%\usepackage{float}
\graphicspath{{./gfx/}}
%\usepackage[font=small,skip=5pt]{caption}

% Setting margins
\usepackage[a4paper,bottom=2cm,top=2cm,left=2.5cm,right=2.5cm, includefoot]{geometry}

% Table helper packages
%\usepackage{multirow, multicol}
%\usepackage{makecell}
%\usepackage{array}
%\usepackage{tabularx} % Not needed currently, but has a few nice options
%\usepackage{wrapfig} % Floating figures/tables
%\usepackage{booktabs}
%\usepackage{catchfile}

% Prevents spamming tedious newlines everywhere, also disables auto indentation, etc.
\usepackage[skip=0.75\baselineskip plus 2pt]{parskip}

% Self-explanatory
%\usepackage{titlesec}
%\titleformat{\section}[block]{\normalfont\scshape\Large}{\thesection}{1em}{}
%\titleformat{\subsection}{\normalfont\large}{\thesubsection}{1em}{}

% Plotting package
\usepackage{pgfplots}
\pgfplotsset{compat = newest}

\begin{document}
  \begin{center}
    {\Large CS 342 - Operating Systems}\\
    {\vspace{1.5em}\large Homework \#1}\\
    {Vedat Eren Arıcan - 22002643}
  \end{center}
  
  \section{Installing Linux}

  I installed Ubuntu Server 22.04 on a QEMU virtual machine running on my Arch Linux system.
  From my investigations, I found that there is minimal difference between the Desktop and Server flavors
  for the requirements of this course. Both use the same kernel and the same kernel version,
  and also share the same package repository. The most significant difference is that the server flavor
  excludes bloat GUI packages such as the X server and a desktop environment, both of which can still
  be installed on demand, but aren't vital to me at the moment.
  I already have a personalized Linux desktop environment on my host machine.

  As for the installation experience: I downloaded the ISO image through the official torrents.
  I created a 15 GB raw disk image for QEMU. Then after mounting the installation image,
  the rest of the process was straightforward with the provided menu.

  At any rate, it is trivial to install Ubuntu at any time if needed.

  \subsection{10 Linux Commands}

  \begin{itemize}
    \item \texttt{ls}: List directory
    \item \texttt{man}: Show the manual page for given concept
    \item \texttt{less}: A terminal pager to easily navigate outputs
    \item \texttt{passwd}: Allows modifying user password
    \item \texttt{sudo}: Allows running commands as another user
    \item \texttt{modprobe}: Add/remove modules from the Linux kernel
    \item \texttt{systemctl}: Control the system through the main \textit{systemd} utility.
    \item \texttt{cat}: Concatenates files and prints onto standard output
    \item \texttt{curl}: Tool for data transmission between hosts
    \item \texttt{useradd}: Utility to interact with user information
  \end{itemize}

  \section{Linux Kernel Location}

  The kernel executable resides at \texttt{/boot/vmlinuz-5.15.0-43-generic} on Ubuntu 22.04.

  The output of \texttt{uname -r} is \texttt{5.15.0-43-generic}.

  \section{Linux Kernel Source}

  Minus the hidden files, the root directory of the kernel source contains the directories below:

  \begin{verbatim}
arch/    Documentation/  init/    LICENSES/  scripts/   usr/
block/   drivers/        ipc/     mm/        security/  virt/
certs/   fs/             kernel/  net/       sound/
crypto/  include/        lib/     samples/   tools/
  \end{verbatim}

  For the 64-bit x86 architecture, the system call table is located at:\\
  \texttt{arch/x86/entry/syscalls/syscall\_64.tbl}

  The requested system call names are:

  \begin{itemize}
    \item \textbf{0:} read
    \item \textbf{1:} write
    \item \textbf{2:} open
    \item \textbf{3:} close
    \item \textbf{35:} nanosleep
    \item \textbf{110:} getppid
    \item \textbf{210:} io\_cancel
  \end{itemize}

  \section{The \texttt{strace} Utility}

  Output for \texttt{man}:

  \lstinputlisting[breaklines,basicstyle=\tiny]{dump/man.txt}

  Output for \texttt{ls}:

  \lstinputlisting[breaklines,basicstyle=\tiny]{dump/ls.txt}

  Output for \texttt{touch}:
  
  \lstinputlisting[breaklines,basicstyle=\tiny]{dump/touch.txt}

  \section{The \texttt{time} Utility}

  The following are three distinct commands measured by \texttt{time}:\\
  \verb+du . 0.03s user 0.22s system 45% cpu 0.549 total+\\
  \verb+man time 0.19s user 0.10s system 7% cpu 3.991 total+\\
  \verb+curl https://ipecho.net/plain 0.03s user 0.02s system 13% cpu 0.377 total+

  The \textit{real} time is the most self-explanatory: It measures the total time passed
  while executing the program in terms of the clock on the wall.
  The \textit{user} and \textit{sys} are equal in characteristics, but differ in
  the trigger for their measurement.
  They both measure the amount of CPU time spent while executing the process,
  but disregard the time spent waiting for "blocked" calls such as I/O events.
  Their triggers are user-mode code (non-kernel) and kernel code, respectively.

  \section{C Programming}

  Source from \texttt{tree.c}:

  \lstinputlisting[breaklines]{src/tree.c}

  And the \texttt{Makefile}:

  \lstinputlisting[breaklines]{Makefile}
  
\end{document}

